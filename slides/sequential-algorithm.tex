\section*{Sequential algorithm}

\subsection*{Informal description}
\begin{frame}[fragile]{\secname}{\subsecname}
\begin{itemize}
\item<1-> Process the vertices of the mesh by \textbf{increasing} value of $f$.
\item<2-> Construct the Reeb graph $\mathcal{R}(f)$ incrementally.
\item<3-> While sweeping upwards, keep:
\begin{itemize}
\item<3-> the \textbf{partial Reeb graph} constructed so far;
\item<4-> the current \sidenotehighlight<1>{\textbf{level set}}{\draw node[anchor=north west,text width=3.7cm] (target) at (.5,-.3) {each connected component corresponds to an open edge of the partial Reeb graph} (mymark) to[out=-90,in=170] ([yshift=-7pt]target.north west);} $f^{-1}(r)$.
\end{itemize}
\item<5-> When processing a vertex, \textbf{update} the level set and the Reeb graph accordingly.
\end{itemize}
\begin{center}
\tikzexternalenable
\begin{tikzpicture}
\path (-2,-.1) rectangle (2.7,3.3);
\begin{scope}[small,transform canvas={scale=.5}]
\colorlet{topcolor}{gray!40!white!30!red}
\colorlet{bottomcolor}{gray!40!white!30!blue}
\alt<2-4>{\def\preimageh{14}}{\def\preimageh{16}}
\tikzmath{\hpercent=int(\preimageh/25*100);}
\colorlet{preimagecol}{topcolor!\hpercent!bottomcolor}
\begin{scope}[shift={(-12,0)}]
\draw[black,line width=2pt] (-1.5pt,0) rectangle (1.5pt,25);
\path[decorate,decoration={markings,mark=at position 1 with{\arrow{Latex[line width=1pt,fill=topcolor,scale=3]};}}] (0,0) to node[left,scale=2] {$f$} ($(0,25)+(0,13pt)$);
\fill[top color=topcolor,bottom color=bottomcolor] (-1.5pt,0) rectangle (1.5pt,25);
\end{scope}
\only<1>{\pic{reeb graph={manifold 1,contour,background={\fill[top color=topcolor,bottom color=bottomcolor] (bbox.south west) rectangle (bbox.north east);}}};}
\begin{onlyenv}<2->
\only<2-3>{\colorlet{preimagecol}{black}}
\pic{reeb graph={manifold 1,contour=black!50,plain background=gray!10}};
\begin{scope}
\clip (-5,-5) rectangle (30,\preimageh);
\pic{reeb graph={manifold 1,manifold bg}};
\end{scope}
\pic{reeb graph={manifold 1,
preimage edge/.style={partial edge={#1}{}{absolute slice={\preimageh}{slice}{\pic[preimagecol,line width=1pt,fill=gray]{slice circle 1={.07cm between (slice-2) and (slice-1)}};}}},
preimage disk/.style={partial edge={#1}{draw=black,opacity=.6,line width=2pt}{absolute up to={\preimageh}{\fill[black,opacity=.6] circle(2pt);}}},
/only=<2-4>{preimage edge/.list={7,8,10}},
/only=<3-4>{preimage disk/.list={7,8,10}},
/only=<5->{preimage edge=10,preimage disk=10,large segment up={9}{7}{8}{\preimageh}{black}{preimagecol},partial edge={9}{draw=black,opacity=.6,line width=2pt}{absolute up to={\preimageh}{\fill[black,opacity=.6] circle(2pt);}},some edges={7,8}{draw=black,opacity=.6,line width=2pt}},
/only=<3->{some edges={1,2,3,4,5,6}{draw=black,opacity=.6,line width=2pt},some nodes={1,2,3,4,5,6,7}{\draw[fill=white,line width=1pt] circle(3pt);}},
/only=<5->{node={8}{\draw[fill=white,line width=1pt] circle(3pt);}},
}};
\end{onlyenv}
\end{scope}
\end{tikzpicture}
\tikzexternaldisable
\end{center}
\end{frame}

\subsection*{The preimage graph}
\begin{frame}[fragile]{\secname}{\subsecname}
The level set $f^{-1}(r)$ can be represented by an abstract \textbf{graph} $G_r$: % if it does not contain any vertices
\begin{columns}[T,onlytextwidth]
\begin{column}{.75\textwidth}
\begin{itemize}
\item<2-> \textcolor{orange!60!yellow}{\textbf{nodes}} $\leadsto$ edges of the mesh $\mathcal{M}$;
\item<3-> \textcolor{violet}{\textbf{edges}} $\leadsto$ \setsidenotecolor[violet]\sidenotehighlight<3>{triangles}{\draw node[anchor=north west,text width=3.5cm] (target) at (.5,-.3) {a triangle connects its two sides intersecting $f^{-1}(r)$} (mymark) to[out=-90,in=170] ([yshift=-7pt]target.north west);}\setsidenotecolor{} of $\mathcal{M}$ intersecting $f^{-1}(r)$.
\end{itemize}
\end{column}
\begin{column}{.25\textwidth}
\tikzexternalenable
\begin{tikzpicture}[scale=.7,y={(0,.5)},line join=bevel]
\pgfmathsetseed{42}
\foreach \i in {0,...,5} {
\coordinate (up-\i) at ($(-\i*36:1)+(rnd*360:rnd*.2)$);
}
\foreach \i in {0,...,4} {
\coordinate (down-\i) at ($(-\i*36-18:1)+(rnd*360:rnd*.2)+(0,-2)$);
}
\fill[violet,opacity=.12,/visible on=<3->] (up-0) \foreach \i in {1,...,5} {-- (up-\i)} -- (down-4) \foreach \i in {3,2,1,0} {-- (down-\i)} -- cycle;
\draw[gray] (up-0) \foreach \i in {1,...,5} {-- (up-\i)} (down-0) \foreach \i in {1,...,4} {-- (down-\i)} (up-0) \foreach \i[evaluate=\i as \j using \i+1] in {0,...,4} {-- (down-\i) -- (up-\j)};
\draw[orange!60!yellow,thick,densely dotted,/visible on=<2->] (up-0) \foreach \i[evaluate=\i as \j using \i+1] in {0,...,4} {-- (down-\i) -- (up-\j)};
\draw[violet,thick,/visible on=<3->] ($.5*(up-0)+.5*(down-0)$) -- ($.5*(up-1)+.5*(down-0)$) \foreach \i[evaluate=\i as \j using \i+1] in {0,...,4} { -- ($.5*(up-\i)+.5*(down-\i)$) -- ($.5*(up-\j)+.5*(down-\i)$)};
\foreach \i[evaluate=\i as \j using \i+1] in {0,...,4} {\draw[orange!60!yellow,fill=white,/visible on=<2->] ($.5*(up-\i)+.5*(down-\i)$) circle(1pt) ($.5*(up-\j)+.5*(down-\i)$) circle (1pt);}
\end{tikzpicture}
\tikzexternaldisable
\end{column}
\end{columns}
\begin{block}<4->{Updating $G_r$}
\begin{itemize}
\item<5-> \textbf{Trigger}: \sidenotehighlight<5>{update}{\draw node[anchor=north west] (target) at (.5,-.3) {from $r=f(v)-\epsilon$ to $r=f(v)+\epsilon$} (mymark) to[out=-90,in=170] ([yshift=-7pt]target.north west);} when processing a vertex \textcolor{blue}{$v$}.\\
\item<6->\textbf{Action}: process each triangle \textcolor{violet!50}{$\mathcal{T}$} of $\Star(\textcolor{blue}{v})$ separately.
\vspace{-.35cm}
\begin{multicols}{2}
\begin{enumerate}
\item<7-> \textcolor{blue}{$v$} is the lower vertex of \textcolor{violet!50}{$\mathcal{T}$}.
\item<11-> \textcolor{blue}{$v$} is the middle vertex of \textcolor{violet!50}{$\mathcal{T}$}.
\item<15-> \textcolor{blue}{$v$} is the upper vertex of \textcolor{violet!50}{$\mathcal{T}$}.
\end{enumerate}
\columnbreak
\tikzexternalenable
\begin{tikzpicture}[scale=.7,
pics/triangle/.style n args={6}{code={
\tikzmath{\j=int(#1+1);}
\scoped[on background layer]\fill[violet,opacity=.12,/visible on=<#1-\j>] (#2) -- (#3) -- (#4) -- cycle;
\alt<-#1>{\draw[violet,thick] #5;}{\draw[violet,thick] #6;}
}}]
\pgfmathsetseed{42}
\coordinate (0) at (0,0);
\foreach \i in {1,...,6} {\coordinate (\i) at ($(60*\i-30:.9)+(rnd*360:rnd*.2)$);}
\foreach \i[evaluate=\i as \j using int(\i+1)] in {0,...,5} \foreach \k in {\j,...,6} {
\coordinate (\i\k) at ($.5*(\i)+.5*(\k)$);
}
\draw[gray] (1) -- (2) -- (3) -- (4) -- (5) -- (6) -- (1) \foreach \i in {1,...,6} {(0) -- (\i)};
\scoped[shift=(16),draw=violet,thick]\draw (0,0) -- (rand*30:.3);
\scoped[shift=(34),draw=violet,thick]\draw (0,0) -- (180+rand*30:.3);
\pic{triangle={7}{0}{4}{5}{(04) -- (05)}{}};
\pic{triangle={9}{0}{5}{6}{(05) -- (06)}{}};
\pic{triangle={11}{0}{3}{4}{(34) -- (04)}{(34) -- (03)}};
\pic{triangle={13}{0}{1}{6}{(16) -- (06)}{(16) -- (01)}};
\pic{triangle={15}{0}{2}{3}{}{(02) -- (03)}};
\pic{triangle={17}{0}{1}{2}{}{(01) -- (02)}};
\foreach \i in {01,02,03,04,05,06,12,23,34,45,56,16} {\draw[orange!60!yellow,fill=white] (\i) circle(1pt);}
\fill[blue] (0) circle(1.5pt);
\end{tikzpicture}
\tikzexternaldisable
\end{multicols}
\vspace{-.4cm}
\item<19-> \textbf{Data structure}: the following operations are required;
\begin{itemize}
\item<20-> \texttt{find} the connected component of a node $e$;
\item<21-> \texttt{insert} a new edge between nodes $e_1$, $e_2$;
\item<22-> \texttt{delete} the edge between nodes $e_1$, $e_2$;
\end{itemize}
\onslide<23->{
$\leadsto$ \sidenotehighlight<23->{offline}{} dynamic connectivity problem \onslide<24->{$\leadsto$ \sidenotehighlight<24>{\textbf{ST-trees}}{\draw node[anchor=north east] (target) at (-.5,-.3) {support all the operations in $O(\log m)$} (mymark) to[out=-90,in=10] ([yshift=-7pt]target.north east);}}}
\end{itemize}
\end{block}
\end{frame}

\subsection*{The augmented Reeb graph}
\begin{frame}[fragile]{\secname}{\subsecname}
The partial augmented Reeb graph is represented by a pair \setsidenotecolor[gray] $(\sidenotehighlight<2>{\mathcal{R}}{\draw node[anchor=south east,align=right] (target) at (0,.3) {Reeb graph constructed so far;\\has one \textcolor{red}{open edge} for each component of \textcolor{violet}{$G_r$}} (mymark) to[out=40,in=-10] ([yshift=-7pt]target.north east);},\sidenotehighlight<3>{\Phi}{\draw node[anchor=north east,align=right] (target) at (-.5,-.3) {partial segmentation map;\\maps each vertex of the mesh to a node or an edge of $\mathcal{R}$} (mymark) to[out=-90,in=10] ([yshift=-7pt]target.north east);})$.\setsidenotecolor{}
\begin{overlayarea}{\textwidth}{.77\textheight}
\begin{onlyenv}<1-3>
\vspace{.6cm}
\begin{center}
\tikzexternalenable
\begin{tikzpicture}
\begin{scope}[small]
\def\preimageh{14}
\pic{reeb graph={manifold 1,/only=<1-3>{contour=black!50,plain background=gray!10},
/utils/exec={\begin{scope}\clip (-5,-1) rectangle (19,\preimageh);},
contour,
/only=<1-2>{plain background=gray!40},
/utils/exec={\end{scope}},
preimage disk/.style={partial edge={#1}{}{absolute slice={\preimageh}{slice}{\getedgecolor{#1}\pic[/alt=<1-2>{violet,line width=1.5pt,fill=gray}{black,line width=1pt,fill=\thiscolor!80!black}]{slice circle 1={.07cm between (slice-2) and (slice-1)}};}}},
preimage edge/.style={partial edge={#1}{/only=<1-2>{/alt=<1>{draw=black,opacity=.6}{draw=red},line width=2pt,densely dashed},/only=<3>{preaction={draw=black,line width=3pt,densely dashed},/utils/exec={\getedgecolor{#1}},draw=\thiscolor,line width=2pt,densely dashed}}{absolute up to={\preimageh}{}}},
/only=<3>{some segments={1,2,3,4,5,6}{\fillsegmentcmd},partial segment={7}{\fillsegmentcmd}{relative down to={0},absolute up to={\preimageh}},partial segment={8}{\fillsegmentcmd}{relative down to={0},absolute up to={\preimageh}},partial segment={10}{\fillsegmentcmd}{relative down to={0},absolute up to={\preimageh}}},
/only=<1-3>{preimage disk/.list={7,8,10},
preimage edge/.list={7,8,10}},
/only=<1-2>{some edges={1,2,3,4,5,6}{draw=black,opacity=.6,line width=2pt}},
/only=<3>{some edges={1,2,3,4,5,6}{preaction={draw=black,line width=3pt},draw=\thiscolor,line width=2pt}},
some nodes={1,2,3,4,5,6,7}{\draw[fill=white,line width=1pt] circle(3pt);},
}};
\end{scope}
\end{tikzpicture}
\tikzexternaldisable
\end{center}
\end{onlyenv}
\begin{onlyenv}<4->
\vspace{-.2cm}
\begin{block}<4->{Updating $(\mathcal{R},\Phi)$}
When processing a vertex $v$:
\begin{enumerate}
\item<5-> \textbf{Let} $\setsidenotecolor[green]\sidenotehighlight<5->{\operatorname{Lc}}{\draw[/visible on=<5>] node[anchor=north west] (target) at (.5,-.3) {lower components} (mymark) to[out=-90,in=170] ([yshift=-7pt]target.north west);}=\left\{G_{f(v)-\epsilon}\texttt{.find}([vv']):v'\in\Link^-(v)\right\}$.
\item<6-> \textbf{Let} $\setsidenotecolor[orange]\sidenotehighlight<6->{\operatorname{Uc}}{\draw[/visible on=<6>] node[anchor=north west] (target) at (.5,-.3) {upper components} (mymark) to[out=-90,in=170] ([yshift=-7pt]target.north west);}=\left\{G_{f(v)+\epsilon}\texttt{.find}([vv']):v'\in\Link^+(v)\right\}$.
\item<7-> \strut\begin{minipage}[t]{.6\textwidth}
\textbf{If} $|\setsidenotecolor[green]\sidenotehighlight<7->{\operatorname{Lc}}{}|=|\setsidenotecolor[orange]\sidenotehighlight<7->{\operatorname{Uc}}{}|=1$ \textbf{then}:\setsidenotecolor{}
\begin{itemize}
\item<8-> $\mathcal{R}$ is unchanged;
\item<9-> $\Phi(v)=$ the \textcolor{red}{open edge} associated to the lower component.
\end{itemize}
\end{minipage}
\begin{minipage}[t]{.3\textwidth}
\begin{center}
\tikzexternalenable
\begin{tikzpicture}[x={(.25,0)},y={(0,.25)},baseline={(0,1.5)}]
\clip (-4,1.2) rectangle (4,5.8);
\pic{reeb graph={
	line width=1pt,
	nodes={
	down={(0,0)},
	up={(2,7)}
	},
	add edge={(1) to[out=170,in=190] (2)},
	contour=black!50,
	plain background=gray!10,
    /alt=<7>{small segment up={1}{.37}{black}{green}}{small segment up={1}{.63}{black}{orange}},
	partial edge={1}{}{relative slice={.37}{slicel}{\pic[green,line width=1pt,opacity=.5,/visible on=<8->]{slice circle 1={.07cm between (slicel-2) and (slicel-1)}};},relative slice={.67}{sliceu}{\pic[orange,line width=1pt,opacity=.5,/visible on=<7>]{slice circle 1={.07cm between (sliceu-2) and (sliceu-1)}};}},
	partial edge={1}{draw=red,line width=1pt,densely dotted}{/alt=<7>{relative up to={.37}{}}{relative up to={.63}{}}},
	/utils/exec={\fill[/alt=<7-8>{black}{red}] (-2,3.5) circle(1.5pt);},
}};
\end{tikzpicture}
\tikzexternaldisable
\end{center}\end{minipage}
\item<10-> \strut\begin{minipage}[t]{.6\textwidth}\textbf{Otherwise}:
\begin{itemize}
\item<11-> create a new node $\textcolor{blue}{w}$ in $\mathcal{R}$;
\item<12-> all the {\only<12>{\color{red}}open edges} associated to the lower components end at $\textcolor{blue}{w}$;
\item<13-> open a \textcolor{red}{new edge} in $\mathcal{R}$ starting at $\textcolor{blue}{w}$ for each upper component.
\item<14-> $\Phi(v)=\textcolor{blue}{w}$.
\end{itemize}
\end{minipage}
\begin{minipage}[t]{.3\textwidth}
\begin{center}
\only<10->{
\tikzexternalenable
\begin{tikzpicture}[x={(.25,0)},y={(0,.25)},baseline={(0,3.6)}]
\alt<-11>{\def\preimageh{7.4}}{\def\preimageh{8.8}}
\clip (-1.5,5.5) rectangle (9,13);
\pic{reeb graph={
	line width=1pt,
	nodes={
		up={(0,0)},
		up={(11,2)},
		up={(6,5)},
		down={(3,8)},
		up={(11,10)},
		up={(16,6)},
		down={(9,13)},
		up={(5,15)},
	},
	add edge={(1) to[out=90,in=190] (3)},
	add edge={(2) to[out=90,in=-10] (3)},
	add edge={(3) to[out=90,in=-90] (4)},
	add edge={(4) to[out=10,in=190] (5)},
	add edge={(4) to[out=170,in=190] (8)},
	contour=black!50,
	plain background=gray!10,
	/utils/exec={\begin{scope}\clip (-5,-5) rectangle (10,\preimageh);},
	manifold bg,
	/utils/exec={\end{scope}},
	partial edge={4}{}{absolute slice={7.4}{slice-r}{}},
	partial edge={5}{}{absolute slice={7.4}{slice-l}{\pic[green,line width=1pt,/alt=<-11>{fill=gray}{opacity=.5}]{slice circle 1=.07cm between (slice-r-2) and (slice-l-1)};}},
    partial edge={4}{}{absolute slice={8.8}{slice}{\pic[orange,line width=1pt,/alt=<-11>{opacity=.5}{fill=gray}]{slice circle 1=.07cm between (slice-2) and (slice-1)};}},
    partial edge={5}{}{absolute slice={8.8}{slice}{\pic[orange,line width=1pt,/alt=<-11>{opacity=.5}{fill=gray}]{slice circle 1=.07cm between (slice-2) and (slice-1)};}},
    /only=<10-11>{partial edge={3}{draw=red,line width=1pt,densely dotted}{absolute up to={\preimageh}{}}},
    /only=<12->{edge={3}{/alt=<12>{draw=red}{draw=black,opacity=.6},line width=1pt}},
    /only=<13->{partial edge={4}{draw=red,line width=1pt,densely dotted}{absolute up to={\preimageh}{}},partial edge={5}{draw=red,line width=1pt,densely dotted}{absolute up to={\preimageh}{}}},
	node={4}{\path[/alt=<10>{fill=black}{draw=blue,fill=white,line width=1pt}] circle(1.5pt);},
}};
\end{tikzpicture}
\tikzexternaldisable
}
\end{center}\end{minipage}
\end{enumerate}
\end{block}
\end{onlyenv}
\end{overlayarea}
\end{frame}

\subsection*{Full implementation}
\begin{frame*}
\begin{algorithm}[H]
\DontPrintSemicolon
\SetKwInOut{Input}{input}\SetKwInOut{Output}{output}
\SetKwFunction{GetLowerComponents}{GetLowerComponents}
\SetKwFunction{GetUpperComponents}{GetUpperComponents}
\SetKwFunction{UpdatePreimageGraph}{UpdatePreimageGraph}
\SetKwFunction{UpdateReebGraph}{UpdateReebGraph}

\Input{a triangulated mesh $\mathcal{M}$\newline a scalar field $f$ on $\mathcal{M}$}
\Output{the augmented Reeb graph $(\mathcal{R},\Phi)$}
\Begin{
$\mathcal{R}$, $\Phi$ $\leftarrow$ $\emptyset$ \textcolor{gray}{[graph]}, $\emptyset$ \textcolor{gray}{[function]}\;
$G_r$ $\leftarrow$ $\emptyset$ \textcolor{gray}{[ST-tree]}\;
sort the vertices of $\mathcal{M}$ by increasing value of $f$\;
\ForEach{$v$ vertex of $\mathcal{M}$}{
    $\operatorname{Lc}$ $\leftarrow$ \GetLowerComponents{$v$}\;
    \UpdatePreimageGraph{}\;
    $\operatorname{Uc}$ $\leftarrow$ \GetUpperComponents{$v$}\;
    \lIf{$|\operatorname{Lc}|=|\operatorname{Uc}|=1$}{update $\Phi(v)$}
    \lElse{\UpdateReebGraph{$v$, $\operatorname{Lc}$, $\operatorname{Uc}$}}
}
\Return{$(\mathcal{R},\Phi)$}
}
\end{algorithm}
\end{frame*}