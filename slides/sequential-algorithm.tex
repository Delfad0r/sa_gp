\section*{Sequential algorithm}

\subsection*{Informal description}
\begin{frame}[fragile]{\secname}{\subsecname}
\begin{itemize}
\item<1-> Process the vertices of the mesh by \textbf{increasing} value of $f$.
\item<2-> Construct the Reeb graph $\mathcal{R}(f)$ incrementally.
\item<3-> While sweeping upwards, keep:
\begin{itemize}
\item<3-> the \textbf{partial Reeb graph} constructed so far;
\item<4-> the current \sidenotehighlight<1>{\textbf{level set}}{\draw node[anchor=north west,text width=3.7cm] (target) at (.5,-.3) {each connected component corresponds to an open edge of the partial Reeb graph} (mymark) to[out=-90,in=170] ([yshift=-7pt]target.north west);} $f^{-1}(r)$.
\end{itemize}
\item<5-> When processing a vertex, \textbf{update} the level set and the Reeb graph accordingly.
\end{itemize}
\begin{center}
\tikzexternalenable
\begin{tikzpicture}
\path (-2,-.1) rectangle (2.7,3.3);
\begin{scope}[small,transform canvas={scale=.5}]
\colorlet{topcolor}{gray!40!white!30!red}
\colorlet{bottomcolor}{gray!40!white!30!blue}
\alt<2-4>{\def\preimageh{14}}{\def\preimageh{16}}
\tikzmath{\hpercent=int(\preimageh/25*100);}
\colorlet{preimagecol}{topcolor!\hpercent!bottomcolor}
\begin{scope}[shift={(-12,0)}]
\draw[black,line width=2pt] (-1.5pt,0) rectangle (1.5pt,25);
\path[decorate,decoration={markings,mark=at position 1 with{\arrow{Latex[line width=1pt,fill=topcolor,scale=3]};}}] (0,0) to node[left,scale=2] {$f$} ($(0,25)+(0,13pt)$);
\fill[top color=topcolor,bottom color=bottomcolor] (-1.5pt,0) rectangle (1.5pt,25);
\end{scope}
\only<1>{\pic{reeb graph={manifold 1,contour,background={\fill[top color=topcolor,bottom color=bottomcolor] (bbox.south west) rectangle (bbox.north east);}}};}
\begin{onlyenv}<2->
\only<2-3>{\colorlet{preimagecol}{black}}
\pic{reeb graph={manifold 1,contour=black!50,plain background=gray!10}};
\begin{scope}
\clip (-5,-5) rectangle (30,\preimageh);
\pic{reeb graph={manifold 1,manifold bg}};
\end{scope}
\pic{reeb graph={manifold 1,
preimage edge/.style={partial edge={#1}{}{absolute slice={\preimageh}{slice}{\pic[preimagecol,line width=1pt,fill=gray]{slice circle 1={.07cm between (slice-2) and (slice-1)}};}}},
preimage disk/.style={partial edge={#1}{draw=black,opacity=.6,line width=2pt}{absolute up to={\preimageh}{\fill[black,opacity=.6] circle(2pt);}}},
/only=<2-4>{preimage edge/.list={7,8,10}},
/only=<3-4>{preimage disk/.list={7,8,10}},
/only=<5->{preimage edge=10,preimage disk=10,large segment up={9}{7}{8}{\preimageh}{black}{preimagecol},partial edge={9}{draw=black,opacity=.6,line width=2pt}{absolute up to={\preimageh}{\fill[black,opacity=.6] circle(2pt);}},some edges={7,8}{draw=black,opacity=.6,line width=2pt}},
/only=<3->{some edges={1,2,3,4,5,6}{draw=black,opacity=.6,line width=2pt},some nodes={1,2,3,4,5,6,7}{\draw[fill=white,line width=1pt] circle(3pt);}},
/only=<5->{node={8}{\draw[fill=white,line width=1pt] circle(3pt);}},
}};
\end{onlyenv}
\end{scope}
\end{tikzpicture}
\tikzexternaldisable
\end{center}
\end{frame}
\end{document}

\subsection*{The preimage graph}
\begin{frame*}
The level set $f^{-1}(r)$ can be represented by an abstract \textbf{graph} $G_r$: % if it does not contain any vertices
\begin{itemize}
\item \textbf{nodes} $\leadsto$ edges of the mesh $\mathcal{M}$;
\item \textbf{edges} $\leadsto$ \sidenotehighlight{triangles}{\draw node[anchor=north west,text width=3.5cm] (target) at (.5,-.3) {a triangle connects its two sides intersecting $f^{-1}(r)$} (mymark) to[out=-90,in=170] ([yshift=-7pt]target.north west);} of $\mathcal{M}$ intersecting $f^{-1}(r)$.
\end{itemize}
\begin{block}{Updating $G_r$}
\begin{itemize}
\item \textbf{Trigger}: \sidenotehighlight{update}{\draw node[anchor=north west] (target) at (.5,-.3) {from $r=f(v)-\epsilon$ to $r=f(v)+\epsilon$} (mymark) to[out=-90,in=170] ([yshift=-7pt]target.north west);} when processing a vertex $v$\\
\item \textbf{Action}: process each triangle $\mathcal{T}$ of $\Star(v)$ separately.
\begin{enumerate}
\item $v$ is the lower vertex of $\mathcal{T}$.
\item $v$ is the middle vertex of $\mathcal{T}$.
\item $v$ is the upper vertex of $\mathcal{T}$.
\end{enumerate}
\item \textbf{Data structure}: the following operations are required;
\begin{itemize}
\item \texttt{find} the connected component of a node $e$;
\item \texttt{insert} a new edge between nodes $e_1$, $e_2$;
\item \texttt{delete} the edge between nodes $e_1$, $e_2$;
\end{itemize}
$\leadsto$ \sidenotehighlight{offline}{} dynamic connectivity problem $\leadsto$ \sidenotehighlight{\textbf{ST-trees}}{\draw node[anchor=north east] (target) at (-.5,-.3) {support all the operations in $O(\log m)$} (mymark) to[out=-90,in=10] ([yshift=-7pt]target.north east);}
\end{itemize}
\end{block}
\end{frame*}

\subsection*{The augmented Reeb graph}
\begin{frame*}
The partial augmented Reeb graph is represented by a pair $(\sidenotehighlight{\mathcal{R}}{\draw node[anchor=south east,align=right] (target) at (-.5,.3) {Reeb graph constructed so far;\\has one open edge for each component of $G_r$} (mymark) to[out=90,in=-10] ([yshift=-7pt]target.north east);},\sidenotehighlight{\Phi}{\draw node[anchor=north east,align=right] (target) at (-.5,-.3) {partial segmentation map;\\maps each vertex of the mesh to a node or an edge of $\mathcal{R}$} (mymark) to[out=-90,in=10] ([yshift=-7pt]target.north east);})$.
\begin{block}{Updating $(\mathcal{R},\Phi)$}
When processing a vertex $v$:
\begin{enumerate}
\item \textbf{Let} $\sidenotehighlight{\operatorname{Lc}}{\draw node[anchor=north west] (target) at (.5,-.3) {lower components} (mymark) to[out=-90,in=170] ([yshift=-7pt]target.north west);}=\left\{G_{f(v)-\epsilon}\texttt{.find}([vv']):v'\in\Link^-(v)\right\}$.
\item \textbf{Let} $\sidenotehighlight{\operatorname{Uc}}{\draw node[anchor=north west] (target) at (.5,-.3) {upper components} (mymark) to[out=-90,in=170] ([yshift=-7pt]target.north west);}=\left\{G_{f(v)+\epsilon}\texttt{.find}([vv']):v'\in\Link^+(v)\right\}$.
\item \textbf{If} $|\operatorname{Lc}|=|\operatorname{Uc}|=1$ \textbf{then}:
\begin{itemize}
\item $\mathcal{R}$ is unchanged;
\item $\Phi(v)=\text{the open edge associated to the lower component}$.
\end{itemize}
\item \textbf{Otherwise}:
\begin{itemize}
\item create a new vertex $w$ in $\mathcal{R}$;
\item all the open edges associated to the lower components end at $w$;
\item open a new edge in $\mathcal{R}$ starting at $w$ for each upper component.
\item $\Phi(v)=w$.
\end{itemize}
\end{enumerate}
\end{block}
\end{frame*}

\subsection*{Full implementation}
\begin{frame*}
\begin{algorithm}[H]
\DontPrintSemicolon
\SetKwInOut{Input}{input}\SetKwInOut{Output}{output}
\SetKwFunction{GetLowerComponents}{GetLowerComponents}
\SetKwFunction{GetUpperComponents}{GetUpperComponents}
\SetKwFunction{UpdatePreimageGraph}{UpdatePreimageGraph}
\SetKwFunction{UpdateReebGraph}{UpdateReebGraph}

\Input{a triangulated mesh $\mathcal{M}$\newline a scalar field $f$ on $\mathcal{M}$}
\Output{the augmented Reeb graph $(\mathcal{R},\Phi)$}
\Begin{
$\mathcal{R}$, $\Phi$ $\leftarrow$ $\emptyset$ \textcolor{gray}{[graph]}, $\emptyset$ \textcolor{gray}{[function]}\;
$G_r$ $\leftarrow$ $\emptyset$ \textcolor{gray}{[ST-tree]}\;
sort the vertices of $\mathcal{M}$ by increasing value of $f$\;
\ForEach{$v$ vertex of $\mathcal{M}$}{
    $\operatorname{Lc}$ $\leftarrow$ \GetLowerComponents{$v$}\;
    \UpdatePreimageGraph{}\;
    $\operatorname{Uc}$ $\leftarrow$ \GetUpperComponents{$v$}\;
    \lIf{$|\operatorname{Lc}|=|\operatorname{Uc}|=1$}{update $\Phi(v)$}
    \lElse{\UpdateReebGraph{$v$, $\operatorname{Lc}$, $\operatorname{Uc}$}}
}
\Return{$(\mathcal{R},\Phi)$}
}
\end{algorithm}
\end{frame*}