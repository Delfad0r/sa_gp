\section{Reeb graph}

\subsection{Definition}
\begin{frame*}
Given:
\begin{itemize}
\item<1-> a manifold $\mathcal{M}$;
\item<2-> a Morse function $\map{f}{\mathcal{M}}{\RR}$ with distinct critical values;
\end{itemize}
\onslide<3->{
the \defterm{Reeb graph} of $f$ is the $1$-dimensional simplicial complex
\[
\mathcal{R}(f)=\mathcal{M}/\sidenote<4>[text height=height("I"))]{\sim}{\draw node[draw,circle,inner sep=5pt] (source)  {} node[text width=4cm,anchor=north west] (target) at (.3,-.3) {$x\sim y$ if $f(x)=f(y)$ and they belong to the same connected component of $f^{-1}(f(x))$} (source) to[out=-90,in=170] ([yshift=-7pt]target.north west);}.
\]
}
\onslide<5->{
The \defterm{segmentation map} is the quotient map
\[
\map{\Phi}{\mathcal{M}}{\mathcal{R}(f)}.
\]
}
\vspace{-.9cm}
\begin{center}
\tikzexternalenable
\begin{tikzpicture}
\colorlet{topcolor}{gray!40!white!30!red}
\colorlet{bottomcolor}{gray!40!white!30!blue}
\def\preimageh{19}
\tikzmath{\hpercent=int(\preimageh/25*100);}
\colorlet{preimagecol}{topcolor!\hpercent!bottomcolor}
\path (-2,-.1) rectangle (6.5,3.3);
\begin{scope}[small,transform canvas={scale=.5}]
\begin{onlyenv}<2->
\begin{scope}[shift={(-12,0)}]
\draw[black,line width=2pt] (-1.5pt,0) rectangle (1.5pt,25);
\path[decorate,decoration={markings,mark=at position 1 with{\arrow{Latex[line width=1pt,fill=topcolor,scale=3]};}}] (0,0) to node[left,scale=2] {$f$} ($(0,25)+(0,13pt)$);
\fill[top color=topcolor,bottom color=bottomcolor] (-1.5pt,0) rectangle (1.5pt,25);
\scoped[shift={(0,\preimageh)}]\draw[/visible on=<4>,line width=1pt,fill=preimagecol] coordinate (level set l) (-4pt,-4pt) rectangle (4pt,4pt);
\end{scope}
\end{onlyenv}
\draw[/visible on=<5>,line width=1pt,-{To}] (20,12.5) -- node[above,scale=2] {$\Phi$} (27,12.5);
\alt<2>{\pic{reeb graph={manifold 1,contour,background={\fill[top color=topcolor,bottom color=bottomcolor] (bbox.south west) rectangle (bbox.north east);}}};}
{\pic{reeb graph={manifold 1,manifold bg,/only=<4>{
partial edge={12}{}{absolute slice={\preimageh}{slice}{\pic[preimagecol,line width=2pt]{slice circle 2=.12cm between (slice-2) and (slice-1)};}},
partial edge={11}{}{absolute slice={\preimageh}{slice}{\pic[preimagecol,line width=2pt]{slice circle 2=.12cm between (slice-2) and (slice-1)};}},
partial edge={10}{}{absolute slice={\preimageh}{slice}{\pic[preimagecol,line width=2pt]{slice circle 2=.12cm between (slice-2) and (slice-1)};}},
},/only=<5>{every segment={\fillsegmentcmd}}}};}
\begin{scope}[xshift=8cm]
\begin{onlyenv}<3->
\pic{reeb graph={manifold 1,every edge={preaction={draw=black,line width=3pt},/alt=<5>{draw=\thiscolor}{draw=gray!40},line width=2pt},every node={\draw[black,line width=1pt,fill=white] circle(3pt);},/only=<4>{
partial edge={12}{}{absolute up to={\preimageh}{\fill[preimagecol] circle(3pt);}},
partial edge={11}{}{absolute up to={\preimageh}{\fill[preimagecol] circle(3pt);}},
partial edge={10}{}{absolute up to={\preimageh}{\fill[preimagecol] coordinate (level set r) circle(3pt);}},}
}};
\end{onlyenv}
\end{scope}
\only<4>{\scoped[on background layer={transform canvas={scale=.5}}]\draw[line width=1,preimagecol,dashed] (level set l) -- (level set r);}
\end{scope}
\end{tikzpicture}
\tikzexternaldisable
\end{center}
\end{frame*}

\subsection*{Desired algorithm}
\begin{frame*}
\begin{columns}[T,onlytextwidth]
\begin{column}{.7\textwidth}
\textbf{Input:}
\begin{itemize}
\item {\only<2->{\color{gray}}a PL manifold $\mathcal{M}$}\\
\onslide<2->{$\leadsto$ a triangulated mesh $\mathcal{M}$;}
\item<3-> {\only<4->{\color{gray}}a non-degenerate PL scalar field $f$ on $\mathcal{M}$}\\
\onslide<4->{$\leadsto$ a \sidenotehighlight<4>{scalar value}{\draw node[anchor=north west,text width=4cm] (target) at (.5, -.3) {pairwise different, in order to ensure non-degeneracy; this can be achieved by random perturbations} (mymark) to[out=-90,in=170] ([yshift=-7pt]target.north west);} $f(v)$ for each vertex $v$ of $\mathcal{M}$.}
\end{itemize}
\vspace{.4cm}
\onslide<5->{
\textbf{Output:}
\begin{itemize}
\item the \sidenotehighlight<5>{augmented}{\draw node[anchor=north west] (target) at (.5,-.3) {graph $+$ segmentation map} (mymark) to[out=-90,in=170] ([yshift=-7pt]target.north west);} Reeb graph $\mathcal{R}(f)$.
\end{itemize}
}
\end{column}
\begin{column}{.3\textwidth}
\tikzexternalenable
\begin{tikzpicture}
\colorlet{topcolor}{gray!40!white!30!red}
\colorlet{bottomcolor}{gray!40!white!30!blue}
\path(0,0) rectangle(3,3.5);
\begin{scope}[small,transform canvas={scale=.5},shift={(6,1)}]
\only<1-2>{\pic{reeb graph={manifold 1,manifold bg}};}
\only<3-4>{\pic{reeb graph={manifold 1,contour,background={\fill[top color=topcolor,bottom color=bottomcolor] (bbox.south west) rectangle (bbox.north east);}}};}
\only<5->{\pic{reeb graph={manifold 1,contour, every segment={\fillsegmentcmd},every edge={preaction={draw=white,line width=3pt},draw=\thiscolor,line width=2pt},every node={\draw[black,line width=1pt,fill=white] circle(3pt);}}};}
\end{scope}
\end{tikzpicture}
\tikzexternaldisable
\end{column}
\end{columns}
\vspace{.4cm}
\onslide<6->{
\textbf{Time complexity:}
\begin{itemize}
\item $O(m\cdot\log m)$, where $m$ is the \sidenotehighlight<6>{size}{\draw node[anchor=north east] (target) at (-.5,-.3) {$\#\text{vertices}+\#\text{edges}+\#\text{triangles}$} (mymark) to[out=-90,in=10] ([yshift=-7pt]target.north east);} of the $2$-skeleton of $\mathcal{M}$.
\end{itemize}
}
\vspace{.4cm}
\onslide<7->{
\textbf{Parallel.}
}
\end{frame*}

\subsection{Geometry of critical points}
\begin{frame*}
There are three kinds of critical points:
\begin{columns}[T,onlytextwidth]
\begin{column}{.55\textwidth}
\begin{itemize}
\item<2-> \textcolor{gray}{(local)} \textbf{maxima}\\
\onslide<9->{$\leadsto$ ${\color{orange}\Link^+}$ empty;}
\item<3-> \textcolor{gray}{(local)} \textbf{minima}\\
\onslide<10->{$\leadsto$ ${\color{violet}\Link^-}$ empty;}
\item<4-> \textbf{saddles}\\
\onslide<11->{$\leadsto$ ${\color{orange}\Link^+}$ or ${\color{violet}\Link^-}$ disconnected.}
\end{itemize}
\end{column}
\begin{column}{.45\textwidth}
\tikzexternalenable
\begin{tikzpicture}
\matrix[cells={scale=.5},row sep=7, row 1/.style={/visible on=<2->},row 2/.style={/visible on=<3->},row 3/.style={/visible on=<4->}]{
\draw[fill=gray!40] (-1,-1) arc(180:0:1) arc(0:-180:1 and .25);
\draw[postaction={/visible on=<10->,draw=violet,line width=1},dashed,opacity=.5,/visible on=<2->] (-1,-1) arc(180:0:1 and .25);
\draw[/visible on=<10->,violet,line width=1] (-1,-1) arc(-180:0:1 and .25);
\draw[fill=white] circle(3pt);\\
\draw[fill=gray!40] (-1,1) arc(-180:0:1) -- cycle;
\draw[postaction={/visible on=<9->,draw=orange,line width=1},fill=gray] (0,1) circle(1 and .25);
\draw[fill=white] circle(3pt);\\
\fill[gray] (-1,.4) arc(90:0:.6 and .3) arc(-180:0:.4 and .1) arc(180:90:.6 and .3) -- (1,0) to[bend left] (-1,0) -- cycle;
\fill[gray!40] (-1,-1) to[bend left] (1,-1) -- (1,-.2) arc(-90:-180:.6 and .3) arc(0:-180:.4 and .1) arc(0:-90:.6 and .3) -- cycle;
\draw[postaction={/visible on=<11->,draw=violet,line width=1}] (-1,-1) to[bend left] (1,-1);
\path[name path=first] (-1,0) to[bend right] (1,0);
\draw[name path=second,postaction={/visible on=<11->,draw=orange,line width=1}] (-1,-.2) arc(-90:90:.6 and .3) (1,-.2) arc(270:90:.6 and .3);
\draw[postaction={/visible on=<11->,draw=violet,line width=1},dashed,opacity=.5,/visible on=<4->,intersection segments={of={first and second}, sequence={L2}}];
\draw[postaction={/visible on=<11->,draw=violet,line width=1},intersection segments={of={first and second}, sequence={L1 L3}}];
\draw (-.4,.1) arc (-180:0:.4 and .1);
\draw[fill=white] circle(3pt);\\
};
\end{tikzpicture}
\tikzexternaldisable
\end{column}
\end{columns}
\onslide<5->{
\begin{block}{How to detect them on a PL manifold?}
\begin{columns}[T,onlytextwidth]
\begin{column}{.65\textwidth}
\onslide<6->{Given a vertex $v$, the {\color{blue!40}\defterm{star}} of $v$ is the union of all simplices containing $v$.\\}
\onslide<7->{The {\color{blue}\defterm{link}} of $v$ is the boundary of its star.}
\onslide<8->{
\begin{align*}
{\color{orange}\Link^+}(v)&=\left\{x\in\Link(v):f(x)>f(v)\right\}\\
{\color{violet}\Link^-}(v)&=\left\{x\in\Link(v):f(x)<f(v)\right\}\\
\end{align*}
}
\end{column}
\begin{column}{.3\textwidth}
\tikzexternalenable
\begin{tikzpicture}[light edge/.style={gray}]
\pgfmathsetseed{42}
\clip (-1.5,-1.3) rectangle(1.5,1.3);
\foreach \i in {0,...,5} {
\scoped[shift={(60*\i+30:1)}]\coordinate (\i) at (rnd*360:rand*.2);
\draw[light edge] (0,0) -- (\i);
}
\foreach \i in {0,...,11} {
\scoped[shift={(30*\i:2)}]\coordinate (o-\i) (rnd*360:rand*.2);
\tikzmath{
\j = div(\i, 2); 
if mod(\i,2) then {{\draw[light edge] (o-\i) --(\j);};}
else {\k=mod(\j+5,6); {\draw[light edge] (\j) -- (o-\i) -- (\k);};};
}}
\draw[/alt=<7>{blue,line width=1}{light edge}] (0) -- (1) -- (2) -- (3) -- (4) -- (5) -- cycle;
\only<8->{
\foreach \ys/\col in {1/orange,-1/violet}{
\begin{scope}
\clip[yscale=\ys] (-2,0) rectangle(2,2);
\draw[\col,line width=1] (0) -- (1) -- (2) -- (3) -- (4) -- (5) -- cycle;
\end{scope}
}}
\draw[fill=white] circle(1.5pt);
\scoped[on background layer]\fill[/visible on=<6>,blue!20] (0) -- (1) -- (2) -- (3) -- (4) -- (5) -- cycle;
\end{tikzpicture}
\tikzexternaldisable
\end{column}
\end{columns}
\end{block}
}
\end{frame*}

\subsection*{Significance of critical points}
\begin{frame}[fragile]{\secname}{\subsecname}
\colorlet{maximum color}{orange!50!yellow}
\colorlet{minimum color}{red}
\colorlet{saddle up color}{green!90!black}
\colorlet{saddle down color}{green!40!blue}
The critical points of $f$ are closely related to the topology of the Reeb graph $\mathcal{R}(f)$.
\begin{itemize}
\item<2-> \textcolor{maximum color}{\textbf{Maxima}} and \textcolor{minimum color}{\textbf{minima}} $\leadsto$ nodes of valence $1$ (leaves).
\item<3-> \textbf{Saddles} $\leadsto$ nodes of valence $\ge 2$.\\
\begin{itemize}
\item<4-> \textcolor{saddle down color}{\textbf{Join saddles}}: multiple components below.\tikzmarknode{prev}{}
\item<4-> \textcolor{saddle up color}{\textbf{Split saddles}}: multiple components above.\sidenote<5>{}{\draw[-,decorate,decoration={brace,mirror}] ($(mymark)+(.3em,-.3em)$) -- node[right,inner xsep=10pt,align=left] {non-mutually exclusive\\in dimension $\ge 3$} ($(mymark |- prev)+(.3em,1em)$);}
\end{itemize}
\end{itemize}
\begin{center}
\tikzexternalenable
\begin{tikzpicture}
\path (-.7,-.1) rectangle (3.1,4.5);
\begin{scope}[small,transform canvas={scale=.7}]
\pic{reeb graph={
minimum/.style={
small segment up={#1}{.2}{minimum color}{minimum color}
},
maximum/.style={
small segment down={#1}{.2}{maximum color}{maximum color}
},
saddle up/.style n args={3}{
small segment up={#2}{.2}{saddle up color},
small segment up={#3}{.2}{saddle up color},
large segment down={#1}{#2}{#3}{.2}{saddle up color}{saddle up color}
},
saddle down/.style n args={3}{
small segment down={#2}{.2}{saddle down color},
small segment down={#3}{.2}{saddle down color},
large segment up={#1}{#2}{#3}{.2}{saddle down color}{saddle down color}
},
manifold 1,
/alt=<1>{manifold bg}{contour=black!50,plain background=gray!10},
/only=<2>{maximum/.list={12,13},minimum/.list={1,2,5}},
/only=<3->{saddle up={3}{7}{4},saddle up={6}{8}{10},saddle up={9}{12}{11},saddle down={3}{1}{2},saddle down={6}{4}{5},saddle down={9}{7}{8},saddle down={13}{11}{10}},
/only=<3>{saddle up={3}{7}{4}},
every edge={draw=black,opacity=.6,line width=1.5pt},
/only=<1>{every node={\draw[line width=1pt,fill=white] circle(2pt);}},
/only=<2>{every node but={1,2,6,11,12}{\draw[black!50,line width=1pt,fill=white] circle(2pt);},some nodes={1,2,6}{\draw[line width=1pt,fill=minimum color] circle(2pt);},some nodes={11,12}{\draw[line width=1pt,fill=maximum color] circle(2pt);}},
/only=<3->{every node but={3,4,5,7,8,9,10}{\draw[black!50,line width=1pt,fill=white] circle(2pt);},some nodes={4,7,9}{\draw[line width=1pt,fill=saddle up color] circle(2pt);},some nodes={3,5,8,10}{\draw[line width=1pt,fill=saddle down color] circle(2pt);}}
}};
\end{scope}
\end{tikzpicture}
\tikzexternaldisable
\end{center}
\end{frame}