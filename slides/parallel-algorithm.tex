\section{Parallel algorithm}

\subsection*{Core ideas}
\begin{frame*}
\begin{itemize}
\item[{\tikz\node[inner sep=0,opacity=.5]{\usebeamertemplate**{itemize item}};}] \textcolor{gray}{\textbf{Sequential}: single procedure sweeping all the vertices sequentially.}
\item \textbf{Parallel}: multiple procedures (local growths) running simultaneously.
\begin{itemize}
\item A local growth is started at every minimum.
\item Each local growth spreads independently with an ordered BFS.
\item Each local growth updates its own preimage graph $G_r$.
\item \textbf{Join saddles}: wait until all the involved local growths have reached the saddle, then join them.
\item \textbf{Split saddles}: the new open edges in $\mathcal{R}(f)$ are handled by the same local growth.
\end{itemize}
\end{itemize}
\end{frame*}

\subsection*{Local growths}
\begin{frame*}
\begin{block}{Data structures}
Each local growth keeps:
\begin{itemize}
\item a \sidenotehighlight<1>{\textbf{Fibonacci heap}}{\draw node[anchor=north west] (target) at (.5,-.3) {can be merged in $O(1)$} (mymark) to[out=-90,in=170] ([yshift=-7pt]target.north west);} $\theta$ to store candidates for the \sidenotehighlight<1>{ordered}{\draw node[anchor=south east,text width=2.5cm,align=right] (target) at (-.5,.3) {candidates are sorted by $f$ value} (mymark) to [out=90,in=-10] ([yshift=-7pt]target.north east);} BFS;
\item an \sidenotehighlight<1>{\textbf{ST-tree}}{\draw node[anchor=north west] (target) at (.5,-.3) {can be merged in $O(1)$} (mymark) to[out=-90,in=170] ([yshift=-7pt]target.north west);} $G_r$ to store the preimage graph. 
\end{itemize}
\end{block}
\begin{block}{Join saddles}
What if a saddle joins components from different local growths?
\begin{itemize}
\item \sidenotehighlight<1>{\textbf{Detection}}{\draw node[anchor=north west,text width=6cm] (target) at (.5,-.3cm-1em) {concretely, update an atomic counter $\text{visitedLower}[v]$ and check whether $\text{visitedLower}[v]=|\Link^-(v)|$} (mymark) to[out=-90,in=170] ([yshift=-7pt]target.north west);}: before processing a vertex $v$, check whether all the vertices in $\Link^-(v)$ have already been visited.
\item \textbf{Stopping}: if not, terminate this local growth.
\item \textbf{Processing}: otherwise, this local growth is in charge of proceeding;
\begin{itemize}
\item join the priority queues ($\theta$) and the preimage graphs ($G_r$) of all local growths terminated at $v$;
\item process $v$ as usual.
\end{itemize}
\end{itemize}
\end{block}
\end{frame*}

\subsection*{Local growth implementation}
\begin{frame*}
\begin{procedure}[H]
\DontPrintSemicolon
\SetKwProg{Procedure}{procedure}{}{end}
\SetKwFunction{LocalGrowth}{LocalGrowth}
\SetKwFunction{join}{join}
\SetKwFunction{add}{add}
\SetKwData{pending}{pending}
\SetKwData{visitedLower}{visitedLower}
\SetKwData{incr}{incr}
\SetKw{Return}{terminate}
\Procedure{\LocalGrowth{$v_0$, $\mathcal{R}$, $\Phi$}}{
    $\theta$, $G_r$ $\leftarrow$ $\{v_0\}$ \textcolor{gray}{[Fibonacci heap]}, $\emptyset$ \textcolor{gray}{[ST-tree]}\;
   \While{\tikzmarknode{while}{$\theta\ne\emptyset$}}{
        $v$ $\leftarrow$ vertex in $\theta$ with minimal $f$ value\;
        %$\incr$ $\leftarrow$ $\left|\{w\in\Link^-(v):\text{$w$ was visited by this local growth}\}\right|$\;
        \tikzmarknode{A}{}\sidenotehighlight<1>{update $\visitedLower[v]$}{\draw node[anchor=west] (target) at ([xshift=1.5cm]while) {add $\left|\left\{w\in\Link^-(v):\text{$w$ visited by this local growth}\right\}\right|$} (mymark) to[out=90,in=190] (target.west);} \;
        \If{$\visitedLower[v]<|\Link^-(v)|$}{
            append $(\theta,G_r)$ to $\pending[v]$\;
            \Return\;
        }\sidenote<1>{}{\coordinate (B) at ($(mymark)+(5.7cm,.8\baselineskip)$);\coordinate (C) at ($(A)+(-2pt,.8\baselineskip)$);\draw[-,dotted,thick] (C)  -- ($(A)+(5.7cm,.8\baselineskip)$) edge[-,solid,decorate,decoration={brace}] node[right,inner xsep=10] {critical section} (B) (B) -- ($(mymark)+(-2pt,.8\baselineskip)$) -- (C);}%
        \ForEach{$(\theta',G_r')\in\pending[v]$}{
            $\theta.\join{$\theta'$}$; $G_r.\join{$G_r'$}$\;
        }
        \sidenotehighlight<1>{process $v$, updating $G_r$, $\mathcal{R}$ and $\Phi$}{\draw node[anchor=north] (target) at (.2,-.7) {just as in the sequential algorithm} (mymark.east) to[out=-10,in=10,looseness=1.5] (target.east);}\;
        add vertices in $\Link^+(v)$ to $\theta$\;
    }
}
\end{procedure}
\end{frame*}

\subsection*{Full implementation}
\begin{frame*}
\begin{algorithm}[H]
\DontPrintSemicolon
\SetKwInOut{Input}{input}\SetKwInOut{Output}{output}
\SetKwFunction{FindMinima}{FindMinima}
\SetKwFunction{LocalGrowth}{LocalGrowth}
\SetKwFor{ForEachParallel}{foreach}{\sidenotehighlight<1>{in parallel}{} do}{end}

\Input{a triangulated mesh $\mathcal{M}$\newline a scalar field $f$ on $\mathcal{M}$}
\Output{the augmented Reeb graph $(\mathcal{R},\Phi)$}
\Begin{
$\mathcal{R}$, $\Phi$ $\leftarrow$ $\emptyset$ \textcolor{gray}{[graph]}, $\emptyset$ \textcolor{gray}{[function]}\;
$V$ $\leftarrow$ \sidenotehighlight<1>{\FindMinima{$\mathcal{M},f$}}{\draw node[anchor=west] (target) at (2.5,0) {easy to run in parallel} (mymark) to (target.west);}\;
\ForEachParallel{$v_0\in V$}{
    start procedure \LocalGrowth{$v_0,\mathcal{R},\Phi$}\;
}
\Return{$(\mathcal{R},\Phi)$}
}
\end{algorithm}
\end{frame*}

\end{document}

\subsection*{Optimizations}
\begin{frame*}
\begin{block}{\sidenote{Laziness}{\draw[shorten <=3pt] node[anchor=west] (target) at (2,0) {decreased number of operations on ST-trees} (mymark.east) to (target.west);}}
\begin{itemize}
\item Instead of inserting and removing edges is $G_r$, keep a \sidenotehighlight{\textbf{history list}}{\draw node[anchor=north east] (target) at (-.7,-.8) {one for each open edge in $\mathcal{R}$} (mymark.south) to[out=-90,in=10] ([yshift=-7pt]target.north east);}.
\item When a \sidenotehighlight{critical point}{\draw node[anchor=north west,text width=4cm] (target) at (.6,-1.1) {requires a pre-processing step to identify the critical points} (mymark.south) to[out=-90,in=170] ([yshift=-7pt]target.north west);} is reached, discard all edges marked as both added and deleted.
\item Perform all the other operations on $G_r$.
\end{itemize}
\end{block}
\begin{block}{\sidenote{Dual sweep}{\draw[shorten <=3pt] node[anchor=west] (target) at (2,0) {increased number of independent local growths} (mymark.east) to (target.west);}}
\begin{itemize}
\item Two kinds of local growths:
\begin{itemize}
\item \textbf{upward local growths} $\leadsto$ as described before;
\item \textbf{downward local growths} $\leadsto$ start from maxima and expand downwards.
\end{itemize}
\item Post-processing step to merge edges that have been created twice.
\end{itemize}
\end{block}
\end{frame*}