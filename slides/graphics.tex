\usepackage{ifthen}
\usepackage{tikz}
\usetikzlibrary{math,calc}
\usetikzlibrary{decorations.pathreplacing,decorations.markings}
\usetikzlibrary{quotes}
\usetikzlibrary{backgrounds}
\usetikzlibrary{arrows.meta}
\usetikzlibrary{intersections}
\usepackage{pgfplots}
\usepgfplotslibrary{fillbetween}

\usetikzlibrary{external}
\tikzexternalize[prefix=figures/]
\tikzexternaldisable

\pgfkeys{
alt/.code args={<#1>#2#3}{\alt<#1>{\pgfkeysalso{#2}}{\pgfkeysalso{#3}}},
only/.code args={<#1>#2}{\only<#1>{\pgfkeysalso{#2}}},
visible on/.style args={<#1>}{/alt=<#1>{}{opacity=0}},
}

\newcommand{\fillsegmentcmd}[1][\thiscolor!40]{\fill[#1] (bbox.south west) rectangle (bbox.north east);}

\tikzset{small/.style={x={(.25,0)},y={(0,.25)}}}
\pgfkeys{
/reeb graph settings/.cd,
manifold bg/.style={contour,plain background={gray!40}},
small segment up/.style n args={4}{
partial edge={#1}{/reeb graph settings/.cd,outline edge={#1}{\thickness+.5*\linewidth}{fill=#3}}{relative up to={#2}{},relative down to={0}{}},
partial edge={#1}{/reeb graph settings/.cd,outline edge={#1}{\thickness-.5*\linewidth}{fill=gray!40}}{relative up to={#2}{},relative down to={0}{},relative slice={#2}{slice}{\pic[line width=1pt,#4,fill=gray]{slice circle 1={.07 cm between (slice-2) and (slice-1)}};}},
},
small segment down/.style n args={4}{
partial edge={#1}{/reeb graph settings/.cd,outline edge={#1}{\thickness+.5*\linewidth}{fill=#3}}{relative down to={-#2}{},relative up to={-0.001}{}},
partial edge={#1}{/reeb graph settings/.cd,outline edge={#1}{\thickness-.5*\linewidth}{fill=gray!40}}{relative down to={-#2}{},relative up to={-0.001}{}},
partial edge={#1}{}{relative slice={-.2}{slice}{\pic[opacity=0,fill opacity=1,fill=gray!40]{slice circle 1={.07cm between (slice-2) and (slice-1)}};\pic[line width=1pt,#4]{slice circle 2={.07 cm between (slice-2) and (slice-1)}};}},
},
large segment up/.style n args={6}{
partial edge={#1}{/reeb graph settings/.cd,outline edge={#2}{\thickness+.5*\linewidth}{fill=#5},outline edge={#3}{\thickness+.5*\linewidth}{fill=#5}}{relative up to={#4}{\coordinate (p);},relative down to={0}{}},
partial edge={#2}{}{/utils/exec={\tikzmath{coordinate \p;\p=(p);}},absolute slice={\py}{slice-l}{}},
partial edge={#3}{}{/utils/exec={\tikzmath{coordinate \p;\p=(p);}},absolute slice={\py}{slice-r}{}},
partial edge={#1}{/reeb graph settings/.cd,outline edge={#2}{\thickness-.5*\linewidth}{fill=gray!40},outline edge={#3}{\thickness-.5*\linewidth}{fill=gray!40}}{relative up to={#4}{},relative down to={0}{}},
/utils/exec={\pic[line width=1,#6,fill=gray]{slice circle 1=.07cm between (slice-r-2) and (slice-l-1)};},},
large segment down/.style n args={6}{
partial edge={#1}{/reeb graph settings/.cd,outline edge={#2}{\thickness+.5*\linewidth}{fill=#5},outline edge={#3}{\thickness+.5*\linewidth}{fill=#5}}{relative down to={-#4}{\coordinate (p);},relative up to={-0.001}{}},
partial edge={#2}{}{/utils/exec={\tikzmath{coordinate \p;\p=(p);}},absolute slice={\py}{slice-l}{}},
partial edge={#3}{}{/utils/exec={\tikzmath{coordinate \p;\p=(p);}},absolute slice={\py}{slice-r}{}},
partial edge={#1}{/reeb graph settings/.cd,outline edge={#2}{\thickness-.5*\linewidth}{fill=gray!40},outline edge={#3}{\thickness-.5*\linewidth}{fill=gray!40}}{relative down to={-#4}{},relative up to={-0.001}{}},
/utils/exec={\pic[opacity=0,fill opacity=1,fill=gray!40]{slice circle 1=.07cm between (slice-r-2) and (slice-l-1)};\pic[line width=1,#6]{slice circle 2=.07cm between (slice-r-2) and (slice-l-1)};},
},
manifold 1/.style={
line width=1pt, thickness=.3cm,
nodes={
		up={(0,0)},
		up={(11,2)},
		up={(6,5)},
		down={(3,8)},
		up={(11,10)},
		up={(16,6)},
		down={(9,13)},
		up={(5,15)},
		down={(4,18)},
		up={(10,20)},
		down={(-2,23)},
		down={(9,25)},
	},
	add edge={(1) to[out=90,in=190] (3)},
	add edge={(2) to[out=90,in=-10] (3)},
	add edge={(3) to[out=90,in=-90] (4)},
	add edge={(4) to[out=10,in=190] (5)},
	add edge={(6) to[out=90,in=-10] (5)},
	add edge={(5) to[out=90,in=-90] (7)},
	add edge={(4) to[out=170,in=190] (8)},
	add edge={(7) to[out=170,in=-10] (8)},
	add edge={(8) to[out=90,in=-90] (9)},
	add edge={(7) to[out=10,in=-10] (10)},
	add edge={(9) to[out=10,in=190] (10)},
	add edge={(9) to[out=170,in=-90] (11)},
	add edge={(10) to[out=90,in=-45] (12)},
}
}

\def\colorlist{"red","lime!80!green","violet","green","orange","cyan","green!80!blue!90!black","brown!90!black","yellow!70!orange","teal","pink","magenta","blue","olive"}
\newcommand{\getedgecolor}[2][\thiscolor]{\tikzmath{#1={\colorlist}[int(#2)-1];}}
\tikzset{reeb graph/.pic={
	\begin{scope}[local bounding box=bbox]
	\pgfkeys{/reeb graph settings/.cd,defaults,#1}
	\end{scope}
}}
\pgfkeys{
/reeb graph settings/.cd,
thickness/.store in=\thickness,
line width/.store in=\linewidth,
defaults/.style={
	thickness=.3cm,
	line width=1pt,
   nodes n/.initial=0,
   edges n/.initial=0,
   /utils/exec={\xdef\usingcenternodes{0}},
},
nodes/.style={#1},
add node/.code n args={3}{
	\pgfkeysalso{nodes n/.get=\n,nodes n/.evaluated={int(\n+1)},nodes n/.get=\n,node \n kind/.initial=#3}
	\coordinate (\n) at #1;
	\coordinate (\n-center) at #2;
},
up/.style={add node={#1}{($#1+(0,\thickness)$)}{0}},
down/.style={add node={#1}{($#1-(0,\thickness)$)}{1}},
edges/.style={add edge/.list={#1}},
add edge/.style={edges n/.get=\m,nodes n/.get=\n,edges n/.evaluated={int(\m+1)},edges n/.get=\m,edge \m/.initial={#1},use actual nodes,/utils/exec={
\path[decorate,decoration={markings,mark=at position 0 with{\coordinate (-start);},mark=at position -.001 with {\coordinate(-end);}}] #1;
\foreach \coord in {start,end} {
	\tikzmath{
	\mindist=100;
	int \i;
	for \i in {1,...,\n}{
		coordinate \p;
		\p=(\i)-(-\coord);
		if \mindist>veclen(\px,\py) then {
			\mindist=veclen(\px,\py);
			{\expandafter\xdef\csname edge \m \coord\endcsname{\i}};
		};
	};
}}},
edge \m start/.initial/.expanded={\csname edge \m start\endcsname},
edge \m end/.initial/.expanded={\csname edge \m end\endcsname},
},
forall nodes/.code 2 args={\pgfkeysalso{nodes n/.get=\numberofnodes}\foreach #1 in {1,...,\numberofnodes} {\pgfkeysalso{#2}}},
forall edges/.code 2 args={\pgfkeysalso{edges n/.get=\numberofedges}\foreach #1 in {1,...,\numberofedges} {\pgfkeysalso{#2}}},
swap all nodes/.style={forall nodes={\i}{swap node=\i}},
swap node/.code={\coordinate (-tmp) at (#1);\coordinate (#1) at (#1-center);\coordinate(#1-center) at (-tmp);},
use center nodes/.code={\ifthenelse{\usingcenternodes=0}{\xdef\usingcenternodes{1}\pgfkeysalso{swap all nodes}}{}},
use actual nodes/.code={\ifthenelse{\usingcenternodes=1}{\xdef\usingcenternodes{0}\pgfkeysalso{swap all nodes}}{}},
outline edge/.code n args={3}{
	\pgfkeysalso{use center nodes}
	\pgfkeysalso{edge #1/.get=\edge}
	\def\enlarge{
		\def\cnt{\quality}
		\xdef\leftpath{(-a)}
		\xdef\rightpath{(-d)}
		\path[decorate,decoration={markings,mark=between positions {1/\cnt} and {(\cnt-1)/\cnt} step {1/\cnt} with {\edef\i{\pgfkeysvalueof{/pgf/decoration/mark info/sequence number}}\coordinate (-left \i) at (0,#2);\xdef\leftpath{\leftpath (-left \i)}\coordinate (-right \i) at (0,{-(#2)});\xdef\rightpath{(-right \i) \rightpath}}}] \edge;
		\xdef\leftpath{\leftpath (-b)}
		\xdef\rightpath{(-c) \rightpath}
		\tikzmath{coordinate \p;\p=(-b)-(-c);\angl1=atan2(\py,\px);\p=(-d)-(-a);\angl2=atan2(\py,\px);}
		\path[#3] (-a) -- plot[smooth] coordinates {\leftpath} arc[start angle=\angl1,delta angle=-180,radius=#2] -- plot[smooth] coordinates {\rightpath} arc[start angle=\angl2,delta angle=-180,radius=#2] -- cycle;
	}
	\path[decorate,decoration={show path construction,
		lineto code={
			\coordinate (-a) at ($(\tikzinputsegmentfirst)!#2!90:(\tikzinputsegmentlast)$);
			\coordinate (-d) at ($(\tikzinputsegmentfirst)!-1!(-a)$);
			\coordinate (-c) at ($(\tikzinputsegmentlast)!#2!90:(\tikzinputsegmentfirst)$);
			\coordinate (-b) at ($(\tikzinputsegmentlast)!-1!(-c)$);
			\enlarge
		},
		curveto code={
			\coordinate (-a) at ($(\tikzinputsegmentfirst)!#2!90:(\tikzinputsegmentsupporta)$);
			\coordinate (-d) at ($(\tikzinputsegmentfirst)!-1!(-a)$);
			\coordinate (-c) at ($(\tikzinputsegmentlast)!#2!90:(\tikzinputsegmentsupportb)$);
			\coordinate (-b) at ($(\tikzinputsegmentlast)!-1!(-c)$);
			\enlarge
		},
	}] \edge;
},
contour/.style={forall edges={\i}{outline edge={\i}{\thickness+.5*\linewidth}{fill=#1}}},contour/.default={black},
plain background/.style={forall edges={\i}{outline edge={\i}{\thickness-.5*\linewidth}{fill=#1}}},
background/.style={forall edges={\i}{/utils/exec={\begin{scope}},outline edge={\i}{\thickness-.5*\linewidth}{clip},/utils/exec={#1\end{scope}}}},
node/.code 2 args={
	\pgfkeysalso{use actual nodes}
	\pic[pics/code={#2}] at (#1) {};
},
some nodes/.code 2 args={\foreach \i in {#1} {\pgfkeysalso{node={\i}{#2}}}},
every node/.style={forall nodes={\i}{node={\i}{#1}}},
every node but/.style 2 args={forall nodes={\i}{node={\i}{\tikzmath{
	\ok=1;
	for \j in {#1}{
		if  \i==\j then {
			\ok=0;
		};
	};
	if \ok==1 then {
		{#2};
	};
}}}},
edge/.code 2 args={
	\pgfkeysalso{use actual nodes,edge #1/.get=\edge}
	\getedgecolor{#1}
	\path[#2] \edge;
},
some edges/.code 2 args={\foreach \i in {#1} {\pgfkeysalso{edge={\i}{#2}}}},
every edge/.style={forall edges={\i}{edge={\i}{#1}}},
every edge but/.style 2 args={forall edges={\i}{/utils/exec={\tikzmath{
	\ok=1;
	for \j in {#1}{
		if  \i==\j then {
			\ok=0;
		};
	};
	if \ok==1 then {
		{\pgfkeysalso{edge={\i}{#2}}};
	};
}}}},
partial edge/.code n args={3}{\gdef\codeToExecuteAfterEdge{}\begin{scope}\pgfkeys{reeb graph settings/.cd,edge #1/.get=\edge,partial edge settings/.cd,edgen=#1,#3}\pgfkeysalso{edge={#1}{#2}}\end{scope}\codeToExecuteAfterEdge},
segment/.style 2 args={partial segment={#1}{#2}{relative up to=-.001,relative down to=0}},
some segments/.code 2 args={\foreach \i in {#1} {\pgfkeysalso{segment={\i}{#2}}}},
every segment/.style={forall edges={\i}{segment={\i}{#1}}},
every segment but/.style 2 args={forall edges={\i}{/utils/exec={\tikzmath{
	\ok=1;
	for \j in {#1}{
		if  \i==\j then {
			\ok=0;
		};
	};
	if \ok==1 then {
		{\pgfkeysalso{segment={\i}{#2}}};
	};
}}}},
partial segment/.code n args={3}{
    \gdef\codeToExecuteAfterEdge{}
	\begin{scope}
	\getedgecolor{#1}
	\pgfkeys{reeb graph settings/.cd,edges n/.get=\m,edge #1/.get=\edge,partial segment settings/.cd,#3}
	\def\dosegment{\begin{scope}\pgfkeysalso{outline edge={\i}{\thickness-.5*\linewidth}{clip}}#2\end{scope}}
	\edef\i{#1}
	\dosegment{}
	\foreach \coord/\nocoord/\kind in {start/end/0,end/start/1}{\pgfkeysalso{edge #1\coord/.get=\st,node \st kind/.get=\k}\ifthenelse{\k=\kind}{\foreach \i in {1,...,\m}{\pgfkeysalso{edge \i \nocoord/.get=\j}\ifthenelse{\j=\st}{\dosegment{}}{}}}{}}
	\end{scope}},
debug/.style={every edge={every to/.style={pos=.5,"\i",\thiscolor}},every node={\node[circle,inner sep=1pt,fill=white,draw]{\i};}},
}
\pgfkeys{
/reeb graph settings/partial edge settings/.cd,
edgen/.initial=0,
get relative partial edge coordinates/.code n args={2}{\pgfkeys{/reeb graph settings/.cd,use actual nodes}\path[decorate,decoration={markings,mark=at position #2 with {\coordinate (#1);}}] \edge;},
get absolute partial edge coordinates/.code n args={2}{\pgfkeys{/reeb graph settings/.cd,use actual nodes}\path[name path=first] \edge;\path[name path=second] (bbox.west|-0,#2) -- (bbox.east|-0,#2);\path[name intersections={of=first and second,by={#1}}];},
clip horizontal partial segment/.code 2 args={\clip (bbox.west|-#1) rectangle (bbox.east|-#2);},
clip horizontal segment up to/.code={\clip (bbox.south west) rectangle (bbox.east|-#1);},
clip horizontal segment down to/.code={\clip (bbox.west|-#1) rectangle (bbox.north east);},
relative up to/.code 2 args={
\pgfkeysalso{get relative partial edge coordinates={-up}{#1},clip horizontal segment up to={-up}}
\expandafter\gdef\expandafter\codeToExecuteAfterEdge\expandafter{\codeToExecuteAfterEdge\pic[pics/code={#2}] at (-up) {};}},
relative down to/.code 2 args={
\pgfkeysalso{get relative partial edge coordinates={-down}{#1},clip horizontal segment down to={-down}}
\expandafter\gdef\expandafter\codeToExecuteAfterEdge\expandafter{\codeToExecuteAfterEdge\pic[pics/code={#2}] at (-down) {};}},
absolute up to/.code 2 args={
\pgfkeysalso{get absolute partial edge coordinates={-up}{#1},clip horizontal segment up to={-up}}
\expandafter\gdef\expandafter\codeToExecuteAfterEdge\expandafter{\codeToExecuteAfterEdge\pic[pics/code={#2}] at (-up) {};}},
absolute down to/.code 2 args={
\pgfkeysalso{get absolute partial edge coordinates={-down}{#1},clip horizontal segment down to={-down}}
\expandafter\gdef\expandafter\codeToExecuteAfterEdge\expandafter{\codeToExecuteAfterEdge\pic[pics/code={#2}] at (-down) {};}},
relative slice/.code n args={3}{
\pgfkeysalso{get relative partial edge coordinates={-slice}{#1},edgen/.get=\edgen,/reeb graph settings/.cd,outline edge={\edgen}{\thickness-.5*\linewidth}{name path=second}}
\path[name path=first] (bbox.south west|--slice) -- (bbox.south east|--slice);
\path[name intersections={of=first and second,sort by=first,name={#2}}];
\expandafter\gdef\expandafter\codeToExecuteAfterEdge\expandafter{\codeToExecuteAfterEdge #3}
\pgfkeysalso{/reeb graph settings/partial edge settings/.cd}
},
absolute slice/.code n args={3}{
\coordinate (-slice) at (0,#1);
\pgfkeysalso{edgen/.get=\edgen,/reeb graph settings/.cd,outline edge={\edgen}{\thickness-.5*\linewidth}{name path=second}}
\path[name path=first] (bbox.south west|--slice) -- (bbox.south east|--slice);
\path[name intersections={of=first and second,sort by=first,name={#2}}];
\expandafter\gdef\expandafter\codeToExecuteAfterEdge\expandafter{\codeToExecuteAfterEdge #3}
\pgfkeysalso{/reeb graph settings/partial edge settings/.cd}
},
}
\pgfkeys{
/reeb graph settings/partial segment settings/.cd,
relative up to/.style={/reeb graph settings/partial edge settings/.cd,relative up to={#1}{},/reeb graph settings/partial segment settings/.cd},
absolute up to/.style={/reeb graph settings/partial edge settings/.cd,absolute up to={#1}{},/reeb graph settings/partial segment settings/.cd},
relative down to/.style={/reeb graph settings/partial edge settings/.cd,relative down to={#1}{},/reeb graph settings/partial segment settings/.cd},
absolute down to/.style={/reeb graph settings/partial edge settings/.cd,absolute down to={#1}{},/reeb graph settings/partial segment settings/.cd},
}

\tikzset{
pics/slice arc/.style args={#1:#2:#3 between #4 and #5}{code={
\tikzmath{
coordinate \p,\q;
\p=#4;
\q=#5;
\r=veclen(\px-\qx,\py-\qy)/2;
}
\draw[pic actions] (\p) arc[start angle=#1,end angle=#2,x radius=\r pt,y radius=#3];
}},
pics/slice circle 1/.style args={#1 between #2 and #3}{slice arc=0:360:#1 between #2 and #3},
pics/slice circle 2/.style args={#1 between #2 and #3}{code={\pic[dashed,opacity=.5]{slice arc=0:180:#1 between #2 and #3};\pic{slice arc=0:-180:#1 between #2 and #3};}},
}